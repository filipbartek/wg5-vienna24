% https://europroofnet.github.io/wg5-vienna24/

\documentclass{ctuslides}

%\setbeameroption{show notes}

\usepackage{appendixnumberbeamer}
\usepackage{csquotes}
\usepackage[pdf]{graphviz}
\usepackage[detect-weight=true]{siunitx}

\usepackage{newverbs}
% https://www.cvut.cz/logo-a-graficky-manual
% Pantone 300: 0065BD
% Pantone 284: 6AADE4
% ctustyle.sty
%\definecolor{ctubluelight}{HTML}{1478C7}
%\definecolor{ctubluedark}{HTML}{0065BD}
%\Verbdef\NodeStylePrediction{fillcolor="#1478C7", fontcolor=white}
% (white + 6AADE4) / 2
\Verbdef\NodeStylePrediction{fillcolor="#B4D6F1"}

% hyperref should be loaded as late as possible.
%\usepackage{hyperref}
\hypersetup{
pdftitle={Learning to Rank in Automatic Theorem Proving},
pdfauthor={Filip Bártek, Martin Suda}
}
%\hypersetup{hidelinks}

% glossaries must be loaded after hyperref
% glossaries must be loaded after hyperref
\usepackage{glossaries}

\newacronym{asp}{ASP}{answer set programming}
\newacronym{atper}{ATPer}{automatic theorem prover}
\newacronym{atping}{ATP}{automatic theorem proving}
\newacronym{cpu}{CPU}{central processing unit}
\newacronym{fol}{FOL}{first-order logic}
\newacronym{mi}{Mi}{megainstruction}
\newacronym{nn}{NN}{neural network}
\newacronym{par}{PAR}{penalized average runtime}
\newacronym{pcs}{PCS}{parameter configuration space}
\newacronym{sat}{SAT}{propositional satisfiability}
\newacronym{smt}{SMT}{satisfiability modulo theories}
\newacronym{tptp}{TPTP}{Thousands of Problems for Theorem Provers}
\newacronym{ucb}{UCB}{upper confidence bound}

% Czech Technical University
\newacronym{ctu}{CTU}{Czech Technical University in Prague}

% CASC
\newacronym{casc}{CASC}{CADE ATP System Competition}
\newacronym{casc-14}{CASC-14}{CADE-14 ATP System Competition}
\newacronym{casc-j11}{CASC-J11}{11th IJCAR ATP System Competition \cite{DBLP:journals/aicom/SutcliffeD23}}
% https://www.tptp.org/CASC/J11/Design.html#Divisions
\newacronym{tfa}{TFA}{Typed (monomorphic) First-order with Arithmetic theorems}
\newacronym{fof}{FOF}{First-Order Form theorems}
\newacronym{fnt}{FNT}{First-order form Non-Theorems}
\newacronym{ueq}{UEQ}{Unit EQuality clause normal form theorems}

\glsdisablehyper

\newcommand{\email}[1]{\href{mailto:#1}{\texttt{#1}}}

\title{Learning to Rank in Automatic Theorem Proving}
\author[\underline{Filip Bártek}, Martin Suda]{\underline{Filip Bártek} (\email{filip.bartek@cvut.cz}), Martin Suda}
\institute{\Acrlong{ctu}}
\year=2024
\month=3
\day=25
\date{\today}
\ctutitlenote{\tiny This research was supported by the Czech Science Foundation grant 24-12759S and
COST Action CA20111 EuroProofNet.}

\DeclareMathOperator{\sigmoid}{sigmoid}
\DeclareMathOperator{\softplus}{softplus}
\DeclareMathOperator{\Bernoulli}{Bernoulli}
\newcommand{\R}{\mathbb{R}}

\newcommand{\source}[1]{\tiny Credit: #1}

\begin{document}

\ctutitleframe

\begin{frame}{Motivation: Clause selection}
\note{Context: Saturation-based automatic theorem proving, Vampire

Clause selection can be reduced to clause ranking.}
\begin{itemize}
\item Goal: Train a \emph{clause selection model}
\begin{itemize}
\item Input: Set of clauses
\item Output (one of):
\begin{enumerate}
\item Labeling of the input clauses (positive, negative)
\item Best of the input clauses
\item \emph{Ranking of the input clauses}
\end{enumerate}
\end{itemize}
\item Training data (one of):
\begin{enumerate}
\item Clauses with labels (positive, negative)
\item Set of proof derivations.
Each proof derivation is a set of clauses with labels (positive, negative).
\item \emph{Pairs of clauses $C_+, C_-$ such that $C_+$ should be selected before $C_-$}
\end{enumerate}
\end{itemize}
\end{frame}

\begin{frame}{Learning to rank: Pairwise approach}
\begin{itemize}
\item Goal: Train a \emph{ranking model}
\begin{itemize}
\item Input: Set of items (samples, documents) $D$
\item Output: Ranking (permutation) over $D$
\end{itemize}
\item Training example:
Pair of items $(i, j)$ such that $i$ is to be ranked higher than $j$% ($i \triangleright j$)
\item Main application domain: Recommender systems
\end{itemize}
\end{frame}

\section{RankNet}

\begin{frame}{RankNet}
\nocite{DBLP:conf/icml/BurgesSRLDHH05}
\begin{columns}
\column{0.5\textwidth}
\digraph[scale=0.5]{RankNet}{
graph [splines=ortho, ranksep=0.25, margin=0];
node [shape=record, fontsize=14, width=0, height=0];
subgraph cluster_input {
label=<Training example "<i>i</i> should be ranked higher than <i>j</i>">;
margin=10;
style=dashed;
rank=same;
SampleI [label=<<i><b>x</b><sub>i</sub></i>|Item <i>i</i>>];
SampleJ [label=<<i><b>x</b><sub>j</sub></i>|Item <i>j</i>>];
Target [label=<<i>T<sub>ij</sub> &isin; [0, 1]</i>|Target probability>];
}
Model [label=<<i>f</i>|Score model (NN)>];
ScoreI [label=<{<i>o<sub>i</sub> = f(<b>x</b><sub>i</sub>)</i>|Score of item <i>i</i>}>];
ScoreJ [label=<{<i>o<sub>j</sub> = f(<b>x</b><sub>j</sub>)</i>|Score of item <i>j</i>}>];
ScorePair [label=<{<i>o<sub>ij</sub> = o<sub>i</sub> &minus; o<sub>j</sub></i>|Score of item pair <i>(i, j)</i>}>];
Prob [label=<{<i>P<sub>ij</sub> = &sigma;(o<sub>ij</sub>)</i>|Estimated probability of "<i>i</i> should be ranked higher than <i>j</i>"}>];
Loss [label=<{<i>C<sub>ij</sub> = BCE(P<sub>ij</sub>, T<sub>ij</sub>) = &minus; T<sub>ij</sub> log P<sub>ij</sub> &minus; (1 &minus; T<sub>ij</sub>) log (1 &minus; P<sub>ij</sub>)</i>|Binary cross-entropy loss}>];
SampleI -> Model [style="invis"];
SampleJ -> Model [style="invis"];
SampleI -> ScoreI;
SampleJ -> ScoreJ;
ScoreI -> ScorePair [penwidth=3];
ScoreJ -> ScorePair [penwidth=3];
Model -> ScoreI [penwidth=3];
Model -> ScoreJ [penwidth=3];
ScorePair -> Prob -> Loss [penwidth=3];
Target -> Loss;
}

\note{$\sigma$ is the logistic function (also known as the sigmoid), an example of a sigmoid function.

$$\sigma(o_{ij}) = \frac{e^{o_{ij}}}{1 + e^{o_{ij}}} = \frac{1}{1 + e^{-o_{ij}}}$$}
\note{$T_{ij} > 0.5$ means:
$i$ should be ranked higher than $j$;% ($i \triangleright j$)
$o_i$ should be greater than $o_j$;
$o_{ij}$ should be greater than 0;
$P_{ij}$ should be greater than $0.5$.

Ranking a set of items amounts to ordering them by their scores $o_i$.

By design, the relation $R$ defined by \enquote{$iRj \iff P_{ij} \geq 0.5 \iff f(\mathbf{x}_i)$} is a total order.

$$P_{ij} \geq 0.5 \iff o_{ij} \geq 0 \iff o_i \geq o_j \iff f(\mathbf{x}_i) \geq f(\mathbf{x}_j)$$

Loss gradient can backpropagate from $C_{ij}$ to $f$.
$C_{ij}$ is differentiable with respect to $f$.
We can use gradient descent to train $f$.

RankNet loss is (binary) cross-entropy of true distribution $\Bernoulli(T_{ij})$ and estimated distribution $\Bernoulli(\sigma(o_{ij}))$.}
\column{0.5\textwidth}
\vspace{3cm}
$$\sigma(o_{ij}) = \frac{e^{o_{ij}}}{1 + e^{o_{ij}}}$$
\end{columns}
\end{frame}

\begin{frame}{RankNet loss as a function of $o_i - o_j$}
\centering
\includegraphics[scale=1, page=4, clip, trim=1.5cm 19cm 12cm 2.5cm]{references/RankNet}

\source{Burges et al. Learning to rank using gradient descent. ICML 2005. \nocite{DBLP:conf/icml/BurgesSRLDHH05}}
\note{Binary cross-entropy: $-T \log P - (1 - T) \log (1 - P)$}
\end{frame}

\begin{frame}{RankNet with $T_{ij} = 1$}
\begin{columns}
\column{0.5\textwidth}
\centering
\digraph[scale=0.5]{RankNet1}{
graph [splines=ortho, ranksep=0.25, margin=0];
node [shape=record, fontsize=14, width=0, height=0];
subgraph cluster_input {
label=<Training example "<i>i</i> should be ranked higher than <i>j</i>">;
margin=10;
style=dashed;
rank=same;
SampleI [label=<<i><b>x</b><sub>i</sub></i>|Item <i>i</i>>];
SampleJ [label=<<i><b>x</b><sub>j</sub></i>|Item <i>j</i>>];
}
Model [label=<<i>f</i>|Score model (NN)>];
ScoreI [label=<{<i>o<sub>i</sub> = f(<b>x</b><sub>i</sub>)</i>|Score of item <i>i</i>}>];
ScoreJ [label=<{<i>o<sub>j</sub> = f(<b>x</b><sub>j</sub>)</i>|Score of item <i>j</i>}>];
ScorePair [label=<{<i>o<sub>ij</sub> = o<sub>i</sub> &minus; o<sub>j</sub></i>|Score of example <i>(i, j)</i>}>];
Prob [label=<{<i>P<sub>ij</sub> = &sigma;(o<sub>ij</sub>)</i>|Estimated probability of "<i>i</i> should be ranked higher than <i>j</i>"}>];
Loss [label=<{<i>C<sub>ij</sub> = BCE(P<sub>ij</sub>, 1) = &minus; log P<sub>ij</sub></i>|Binary cross-entropy loss}>];
SampleI -> Model [style="invis"];
SampleJ -> Model [style="invis"];
SampleI -> ScoreI;
SampleJ -> ScoreJ;
ScoreI -> ScorePair [penwidth=3];
ScoreJ -> ScorePair [penwidth=3];
Model -> ScoreI [penwidth=3];
Model -> ScoreJ [penwidth=3];
ScorePair -> Prob -> Loss [penwidth=3];
}

\column{0.5\textwidth}
\vspace{5cm}
$$C_{ij} = \log (1 + e^{o_j - o_i})$$
\end{columns}
\note{$$C_{ij} = - \log P_{ij} = \log (1 + e^{o_j - o_i})$$

$$o_i \geq o_j \iff P_{ij} \geq 0.5 \iff C_{ij} \leq \log 2 \approx 0.69$$}
\end{frame}

\begin{frame}{Symbol weight recommender}
\nocite{DBLP:conf/lpar/Bartek023}
\begin{columns}
\column{0.5\textwidth}
\centering
\only<1>{\digraph[scale=0.5]{RankNetWeightTrain}{
graph [splines=ortho, ranksep=0.25, margin=0];
node [shape=record, fontsize=14, width=0, height=0, style=filled, fillcolor=white];
% Training example
subgraph cluster_input {
label=<Training example "<i>C<sub>&minus;</sub></i> is worse than <i>C<sub>+</sub></i>">;
margin=10;
style=dashed;
rank=same;
Problem [label=<Input problem>];
SampleI [label=<<i>C<sub>&minus;</sub></i>|Negative clause>];
SampleJ [label=<<i>C<sub>+</sub></i>|Positive clause>];
}
% Neural network
GCN [label=<GNN>, style="filled,rounded", penwidth=3];
Activation [label=<<i>1 + softplus(*)</i>|Output activation>, style="filled,rounded"];
SymbolWeights [label=<<i><b>w</b></i>|Symbol weights>];
Problem -> GCN;
GCN -> Activation -> SymbolWeights [penwidth=3];
%ATP [label=<ATP>, style=rounded];
%SymbolWeights -> ATP;
%Problem -> ATP;
%ATP -> Result;
%
SymbolCountsI [label=<<i><b>x</b><sub>&minus;</sub></i>|Symbol counts>];
SymbolCountsJ [label=<<i><b>x</b><sub>+</sub></i>|Symbol counts>];
Problem -> SymbolCountsI;
Problem -> SymbolCountsJ;
ScoreI [label=<<i>o<sub>&minus;</sub> = <b>x</b><sub>&minus;</sub> &middot; <b>w</b></i>|Weight of clause <i>C<sub>&minus;</sub></i>>];
ScoreJ [label=<<i>o<sub>+</sub> = <b>x</b><sub>+</sub> &middot; <b>w</b></i>|Weight of clause <i>C<sub>+</sub></i>>];
SampleI -> SymbolCountsI -> ScoreI;
SampleJ -> SymbolCountsJ -> ScoreJ;
SymbolWeights -> ScoreI [penwidth=3];
SymbolWeights -> ScoreJ [penwidth=3];
ScoreI -> ScorePair [penwidth=3];
ScoreJ -> ScorePair [penwidth=3];
% RankNet
ScorePair [label=<{<i>o = o<sub>&minus;</sub> &minus; o<sub>+</sub></i>|Score of clause pair <i>(C<sub>&minus;</sub>, C<sub>+</sub>)</i>}>];
Prob [label=<{<i>P = &sigma;(o)</i>|Estimated probability of "<i>C<sub>&minus;</sub></i> is worse than <i>C<sub>+</sub></i>"}>];
Loss [label=<{<i>l = BCE(P, 1) = &minus; log P</i>|Binary cross-entropy loss}>];
ScorePair -> Prob -> Loss [penwidth=3];
}
}
\only<2>{\digraph[scale=0.5]{RankNetWeightPredict}{
graph [splines=ortho, ranksep=0.25, margin=0];
node [shape=record, fontsize=14, width=0, height=0, style=filled, fillcolor=white];
% Training example
subgraph cluster_input {
label=<Training example "<i>C<sub>&minus;</sub></i> is worse than <i>C<sub>+</sub></i>">;
margin=10;
style=dashed;
rank=same;
Problem [label=<Input problem>, \NodeStylePrediction];
SampleI [label=<<i>C<sub>&minus;</sub></i>|Negative clause>];
SampleJ [label=<<i>C<sub>+</sub></i>|Positive clause>];
}
% Neural network
GCN [label=<GNN>, style="filled,rounded", penwidth=3, \NodeStylePrediction];
Activation [label=<<i>1 + softplus(*)</i>|Output activation>, style="filled,rounded", \NodeStylePrediction];
SymbolWeights [label=<<i><b>w</b></i>|Symbol weights>, \NodeStylePrediction];
Problem -> GCN;
GCN -> Activation -> SymbolWeights [penwidth=3];
%ATP [label=<ATP>, style=rounded];
%SymbolWeights -> ATP;
%Problem -> ATP;
%ATP -> Result;
%
SymbolCountsI [label=<<i><b>x</b><sub>&minus;</sub></i>|Symbol counts>];
SymbolCountsJ [label=<<i><b>x</b><sub>+</sub></i>|Symbol counts>];
Problem -> SymbolCountsI;
Problem -> SymbolCountsJ;
ScoreI [label=<<i>o<sub>&minus;</sub> = <b>x</b><sub>&minus;</sub> &middot; <b>w</b></i>|Weight of clause <i>C<sub>&minus;</sub></i>>];
ScoreJ [label=<<i>o<sub>+</sub> = <b>x</b><sub>+</sub> &middot; <b>w</b></i>|Weight of clause <i>C<sub>+</sub></i>>];
SampleI -> SymbolCountsI -> ScoreI;
SampleJ -> SymbolCountsJ -> ScoreJ;
SymbolWeights -> ScoreI [penwidth=3];
SymbolWeights -> ScoreJ [penwidth=3];
ScoreI -> ScorePair [penwidth=3];
ScoreJ -> ScorePair [penwidth=3];
% RankNet
ScorePair [label=<{<i>o = o<sub>&minus;</sub> &minus; o<sub>+</sub></i>|Score of clause pair <i>(C<sub>&minus;</sub>, C<sub>+</sub>)</i>}>];
Prob [label=<{<i>P = &sigma;(o)</i>|Estimated probability of "<i>C<sub>&minus;</sub></i> is worse than <i>C<sub>+</sub></i>"}>];
Loss [label=<{<i>l = BCE(P, 1) = &minus; log P</i>|Binary cross-entropy loss}>];
ScorePair -> Prob -> Loss [penwidth=3];
}
}
\column{0.5\textwidth}
$$\softplus(x) = \log(1 + e^x)$$
\vspace{2.5cm}
\end{columns}

\note{Output activation $1 + \softplus(x) = 1 + \log(1 + e^x)$ ensures that each symbol weight is $\geq 1$.
\begin{itemize}
\item Ensures fairness of the selection (\enquote{no clause waits forever})
\item Empirically superior to $0.1 + \softplus(x)$, $\softplus(x)$ and $x$
\item Conservative modification of the standard symbol weight $1$
\end{itemize}

Proxy task

Bártek, Suda. A GNN-Advised Clause Selection. LPAR 2023. \cite{DBLP:conf/lpar/Bartek023}
\begin{itemize}
\item ATP: Vampire
\item Problems: TPTP FOF and CNF
\item GNN queried as a preprocessing step
\item Symbol weights include 1 variable weight and 1 equality weight.
\item Training data: Positives are proof clauses, negatives are selected non-proof clauses.
\item Test performance: 1494/3149 proofs found. Baseline: 1402. Improvement: \SI{6.6}{\percent}.
\item Observation: Variable weight should be relatively small.
\end{itemize}}
\end{frame}

\begin{frame}{Symbol precedence recommender}
\nocite{DBLP:conf/cade/Bartek021}
\centering
\digraph[scale=0.5]{RankNetPrecedence}{
graph [splines=ortho, ranksep=0.25, margin=0];
node [shape=record, fontsize=14, width=0, height=0, style=filled, fillcolor=white];
%
Problem [label="Input problem", \NodeStylePrediction];
% Neural network
GCN [label="GNN", style="filled,rounded", penwidth=3, \NodeStylePrediction];
SymbolWeights [label=<<i><b>c</b></i>|Symbol costs>, \NodeStylePrediction];
Problem -> GCN;
GCN -> SymbolWeights [penwidth=3];
subgraph cluster_training {
label=<Training on "<i>&rho;</i> is worse than <i>&pi;</i>">;
margin=10;
style=dashed;
% Training example
%subgraph cluster_input {
%label=<Training example "<i>&rho; is worse than &pi;</i>">;
%margin=10;
%style=dashed;
%rank=same;
SampleI [label=<<i>&rho;</i>|Worse precedence>];
SampleJ [label=<<i>&pi;</i>|Better precedence>];
%}
%
ScoreI [label=<{<i>o<sub>&rho;</sub> = &rho;<sup>-1</sup> &middot; <b>c</b></i>|Cost of <i>&rho;</i>}>];
ScoreJ [label=<{<i>o<sub>&pi;</sub> = &pi;<sup>-1</sup> &middot; <b>c</b></i>|Cost of <i>&pi;</i>}>];
SampleI -> ScoreI;
SampleJ -> ScoreJ;
ScoreI -> ScorePair [penwidth=3];
ScoreJ -> ScorePair [penwidth=3];
% RankNet
ScorePair [label=<{<i>o = o<sub>&rho;</sub> &minus; o<sub>&pi;</sub></i>|Score of precedence pair <i>(&rho;, &pi;)</i>}>];
Prob [label=<{<i>P = &sigma;(o)</i>|Estimated probability of "<i>&rho;</i> is worse than <i>&pi;</i>"}>];
Loss [label=<{<i>l = BCE(P, 1) = &minus; log P</i>|Binary cross-entropy loss}>];
ScorePair -> Prob -> Loss [penwidth=3];
}
SymbolWeights -> ScoreI [penwidth=3];
SymbolWeights -> ScoreJ [penwidth=3];
%
subgraph cluster_prediction {
label=<Prediction>;
margin=10;
style=dashed;
Sort [label=<Sort symbols by cost>, style="filled,rounded", \NodeStylePrediction];
Recommended [label=<<i>&pi;<sup>*</sup></i>|Recommended precedence>, \NodeStylePrediction];
Sort -> Recommended;
}
SymbolWeights -> Sort;
}
\note{For sake of simplicity, we are omitting normalization of precedence cost by $Z_n = \frac{2}{n(n+1)}$.

$$\pi^{-1} \cdot c = \sum_{i=1}^n \pi^{-1}(i) \cdot c(i) = \sum_{i=1}^n i \cdot c(\pi(i))$$}


\note{Symbol precedence instantiates (configures) simplification ordering on terms (KBO or LPO),
which in turn constrains the application of generating rules in superposition calculus.

Bártek, Suda. Neural Precedence Recommender. CADE 2021. \cite{DBLP:conf/cade/Bartek021}
\begin{itemize}
\item Validation performance: Val problems: 7648. Mean solved (5 repeats): \num{3951.6}. Improvement over baseline (\texttt{frequency} heuristic): \SI{4.8}{\percent}.
\item Baseline (\texttt{frequency}) prefers symbols with many occurrences.
\end{itemize}}
\end{frame}

\begin{frame}{Conclusion}
Tasks suitable for RankNet:
\begin{itemize}
\item Goal: Rank a set of items or get a top-ranked item
\item Training data: Ranked pairs of items
\end{itemize}
\pause
Future work with RankNet:
\begin{itemize}
\item Clause selection:
\begin{itemize}
\item Train a full NN clause ranking model to be queried at runtime
\item Generalize symbol counting clause weight to a RNN on term structure
\item Optimize symbol weights on problems with a common signature,
use logistic regression instead of gradient descent
\end{itemize}
\item Simplification ordering on terms: Train KBO symbol weight jointly with precedence
\item Stress top-ranked items more when training
\end{itemize}
\pause
\note{Papers:
\begin{itemize}
\item A GNN-Advised Clause Selection. LPAR 2023. [symbol weight recommender]
\item Neural Precedence Recommender. CADE 2021.
\end{itemize}}
\centering
\textbf{Thank you for your attention!}
\end{frame}

\appendix

\begin{frame}[plain]
% https://tex.stackexchange.com/a/178803/202639
\begin{beamercolorbox}[center]{title}
\usebeamerfont{title}Appendix
\end{beamercolorbox}
\end{frame}

\begin{frame}[allowframebreaks]{References}
\bibliographystyle{plain}
\tiny
\bibliography{main}
\end{frame}

\begin{frame}{DirectRanker}
\centering
\includegraphics[scale=0.9, page=5, clip, trim=1.5cm 14cm 1.5cm 1.5cm]{references/DirectRanker}

\source{K{\"{o}}ppel et al. \cite{DBLP:conf/pkdd/KoppelSWPK019}}
\end{frame}

\begin{frame}{RankNet}
Training example: Pair of items $i, j$ and target probability $\bar{P}_{ij}$ of \enquote{$i \triangleright j$}

Loss function:
\begin{itemize}
\item $o_i = f(i)$ \ldots score of item $i$
\item $o_{ij} = o_i - o_j$ \ldots score of pair of items $i, j$
\item $P_{ij} = \sigma(o_{ij}) = \frac{e^{o_{ij}}}{1 + e^{o_{ij}}}$ \ldots predicted probability of \enquote{$i \triangleright j$}
\note{$\sigma$ is the logistic function (also known as the sigmoid), an example of a sigmoid function.

$$\sigma(o_{ij}) = \frac{e^{o_{ij}}}{1 + e^{o_{ij}}} = \frac{1}{1 + e^{-o_{ij}}}$$}
\item $C_{ij} = -\bar{P}_{ij} \log P_{ij} - (1 - \bar{P}_{ij}) \log (1 - P_{ij}) = -\bar{P}_{ij} o_{ij} + \log (1 + e^{o_{ij}})$ \ldots binary cross-entropy loss
\end{itemize}

Properties: reflexive ($o_{ii} = 0$), antisymmetric ($o_{ij} = -o_{ji}$), transitive ($o_{ij} \geq 0 \land o_{jk} \geq 0 \implies o_{ik} \geq 0$)
\end{frame}

\begin{frame}{Symbol precedence recommender: Overview}
\centering
\digraph[scale=0.5]{SymbolPrecedenceRecommender}{
graph [splines=ortho, ranksep=0.25, margin=0];
node [shape=box, fontsize=14, width=0, height=0];
Problem [label="First-order logic problem"];
Graphifier [style=rounded, label="Graph constructor"];
g [label="Graph"];
GCN [style=<rounded,bold>, label="Graph convolutional network"];
SymbolEmbedding [label="Symbol embeddings"];
SymbolCostModel [style=<rounded,bold>, label="Output layer"];
SymbolCost [label="Symbol costs"];
Problem -> Graphifier -> g -> GCN;
GCN -> SymbolEmbedding -> SymbolCostModel -> SymbolCost [penwidth=3];
subgraph cluster_prediction {
	label="Prediction";
	style=dashed;
	margin=5;
	Sort [style=rounded, label="Sort"];
	Precedence [label=<Recommended precedence>];
	Sort -> Precedence;
}
SymbolCost -> Sort;
subgraph cluster_training {
	label="Training";
	style=dashed;
	margin=5;
	{ rank = same;
	pi [label=<Precedence &pi;>];
	rho [label=<Precedence &rho;>];
	}
	LossFunction [style=rounded, label="Loss function"];
	Loss [label="Loss value"];
	pi -> LossFunction:nw;
	rho -> LossFunction:ne;
	LossFunction -> Loss [penwidth=3];
}
SymbolCost -> LossFunction [penwidth=3];
}

\end{frame}

\begin{frame}{Symbol weight recommender: Overview}
\centering
% TODO: Improve the diagram: make it prettier, use colors etc.
\digraph[scale=0.5]{SymbolWeightRecommender}{
graph [splines=ortho, ranksep=0.25, margin=0];
node [shape=box, fontsize=14, width=0, height=0];
Problem [label="Input problem", pos="0,0!"];
subgraph cluster_recommender {
	label="Recommender";
	style=dashed;
	margin=5;
	GCN [style=<rounded>, label="Neural network", pos="0,10!"];
	Activation [style=<rounded>, label=<Output activation<br/><i>1+softplus(*)</i>>];
}
SymbolCost [label=<Symbol weights <i>w(*)</i>>];
Problem -> GCN -> Activation -> SymbolCost;
%
subgraph cluster_evaluation {
	label="Evaluation";
	style=dashed;
	margin=5;
	ATP [style=<rounded>, label=<Automatic theorem prover>];
	Result [label=<Result>];
	ATP -> Result;
}
Problem -> ATP;
SymbolCost -> ATP;
%
subgraph cluster_training {
	label="Training";
	style=dashed;
	margin=5;
	{ rank = same;
	c1 [label=<Clause <i>C<sub>+</sub></i>>];
	c0 [label=<Clause <i>C<sub>&minus;</sub></i>>];
	}
	{ rank = same;
	wc1 [label=<Clause weight <i>W(C<sub>+</sub>)</i>>];
	wc0 [label=<Clause weight <i>W(C<sub>&minus;</sub>)</i>>];
	}
	c1 -> wc1;
	c0 -> wc0;
	Logit [label=<Logit <i>W(C<sub>&minus;</sub>) &minus; W(C<sub>+</sub>)</i>>];
	wc1 -> Logit;
	wc0 -> Logit;
%	Prob [label=<Likelihood <i>p = sigmoid(W(C<sub>&minus;</sub>) &minus; W(C<sub>+</sub>))</i>>];
	Loss [label=<Loss <i>l = &minus; log sigmoid(W(C<sub>&minus;</sub>) &minus; W(C<sub>+</sub>))</i>>];
%	Logit -> Prob -> Loss;
	Logit -> Loss;
}
SymbolCost -> wc1;
SymbolCost -> wc0;
SymbolCost -> c1 [style="invis"];
SymbolCost -> c0 [style="invis"];
}

\note{This is an instance of RankNet. We are learning symbol weights to rank clauses. The clause ranking task is a mere proxy task.

The scale of influence of the recommender is small, since it does not evaluate the passive clauses themselves, but only the symbol weights.
This is compensated by the computational cheapness:
The recommender is only evaluated once in the beginning of the proof search,
and there is virtually no overhead over standard weight-based selection.

Note the output activation function.
It ensures a positive lower bound on the symbol weights.
This in turn ensures fairness of the selection.
Negative symbol weights tend to lead to useless attractors.}
\end{frame}

\end{document}
