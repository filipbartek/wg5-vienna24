% https://europroofnet.github.io/wg5-vienna24/

\documentclass{easychair}

\usepackage{csquotes}
\usepackage{todonotes}

\usepackage[detect-weight=true,binary-units=true]{siunitx}
\newcommand{\pc}[1]{\SI[round-mode=places,round-precision=1]{#1}{\percent}}

% hyperref should be loaded as late as possible.
%\usepackage{hyperref}
\hypersetup{
pdftitle={Learning to Rank in Automatic Theorem Proving},
pdfauthor={Filip Bártek}
}
%\hypersetup{hidelinks}

% glossaries must be loaded after hyperref
% glossaries must be loaded after hyperref
\usepackage{glossaries}

\newacronym{asp}{ASP}{answer set programming}
\newacronym{atper}{ATPer}{automatic theorem prover}
\newacronym{atping}{ATP}{automatic theorem proving}
\newacronym{cpu}{CPU}{central processing unit}
\newacronym{fol}{FOL}{first-order logic}
\newacronym{mi}{Mi}{megainstruction}
\newacronym{nn}{NN}{neural network}
\newacronym{par}{PAR}{penalized average runtime}
\newacronym{pcs}{PCS}{parameter configuration space}
\newacronym{sat}{SAT}{propositional satisfiability}
\newacronym{smt}{SMT}{satisfiability modulo theories}
\newacronym{tptp}{TPTP}{Thousands of Problems for Theorem Provers}
\newacronym{ucb}{UCB}{upper confidence bound}

% Czech Technical University
\newacronym{ctu}{CTU}{Czech Technical University in Prague}

% CASC
\newacronym{casc}{CASC}{CADE ATP System Competition}
\newacronym{casc-14}{CASC-14}{CADE-14 ATP System Competition}
\newacronym{casc-j11}{CASC-J11}{11th IJCAR ATP System Competition \cite{DBLP:journals/aicom/SutcliffeD23}}
% https://www.tptp.org/CASC/J11/Design.html#Divisions
\newacronym{tfa}{TFA}{Typed (monomorphic) First-order with Arithmetic theorems}
\newacronym{fof}{FOF}{First-Order Form theorems}
\newacronym{fnt}{FNT}{First-order form Non-Theorems}
\newacronym{ueq}{UEQ}{Unit EQuality clause normal form theorems}

\glsdisablehyper

\title{Learning to Rank in Automatic Theorem Proving}

\author{Filip Bártek}

\institute{\href{https://www.cvut.cz/}{\acrlong{ctu}}, Czech Republic\\
\email{filip.bartek@cvut.cz}}

\authorrunning{Bártek}
\titlerunning{Learning to Rank in Automatic Theorem Proving}

\begin{document}

\maketitle

In saturation-based \gls{atping},
clause selection is a crucial heuristic decision point.
\Glspl{nn} have been successfully trained to aid the proof search by approximately prioritizing clauses
%that ultimately compound a proof \cite{DBLP:conf/lpar/Bartek023}.
in a way that leads to a quick derivation of a proof \cite{DBLP:conf/lpar/Bartek023,DBLP:conf/cade/JakubuvCOP0U20,DBLP:conf/cade/ChvalovskyJ0U19,DBLP:conf/cade/000121a,DBLP:conf/lpar/ChvalovskyKPU23}.
% Note: DBLP:conf/lpar/ChvalovskyKPU23 is not saturation-based - it is iProver.
Typically, such a \gls{nn} is trained on a set of clauses that were derived in successful proof searches.
% They were actually _activated_, not only derived.
The clauses are labeled: \emph{Positive} clauses have contributed to a proof, while \emph{negative} clauses have not.
A straightforward approach trains a neural classifier of clauses \cite{DBLP:conf/cade/JakubuvCOP0U20,DBLP:conf/cade/000121a}.
We motivate an alternative approach based on classification of clause \emph{pairs} by two observations:

\begin{enumerate}
\item
The training data in the form of labeled clauses
can be interpreted as a specification of a \emph{relative} preference over clauses:
Each positive clause in a proof search is preferred over each negative clause in the same proof search.
Specifying pairwise preference relation over clauses opens up the possibility of using finer-grained training data
that only compares pairs of causes that belonged to a passive set simultaneously --
we no longer have to consider every positive competing against every negative.

\item
The final output of a clause selection heuristic is one clause from a passive set,
rather than a partition of the passive set into positive and negative sets.
Standard clause selection heuristics order the clauses by a heuristic weight
(for example, derivation tree size or number of symbol occurrences in the clause)
and prioritize clauses with the lowest weight.
Similarly, we may prefer our \gls{nn} to assign weights to clauses in such a way that
the clauses with relatively small weights should be selected early.
\end{enumerate}

\emph{Learning to rank} is the machine learning task of training a ranking model --
a system that ranks an arbitrary set of objects (e.g., clauses).
The training data is typically supplied in the form of a partial order on a set of objects.
\emph{RankNet} \cite{DBLP:conf/icml/BurgesSRLDHH05}
%(later generalized to DirectRanker \cite{DBLP:conf/pkdd/KoppelSWPK019})
introduces a design of the loss function and the last layer of the \gls{nn}
that allows training the \gls{nn} to rank arbitrary objects represented by feature vectors.

To train a clause selection heuristic in a RankNet-based approach,
I trained a classifier of \emph{clause pairs} $C_+, C_-$ that estimates whether $C_+$ is more useful than $C_-$
when these two clauses compete for selection in a proof search \cite{DBLP:conf/lpar/Bartek023}.
The \gls{nn} predicts an intermediate weight $w(C)$ for each clause $C$.
$w(C_+) < w(C_-)$ signifies that $C_+$ is estimated to be more useful than $C_-$,
so ranking a set of clauses amounts to sorting the clauses by their predicted weights.

This design allowed me to define $w$ as a weighted sum of symbol weights constrained to be greater than 1.
The resulting clause weight was a pragmatic conservative modification of the popular symbol counting clause weight.
Defining clause weight in this way would be non-trivial if the \gls{nn} was trained as a clause classifier,
while the RankNet design accommodates such definition naturally.

%I used the trained clause weights to guide proof search in Vampire, solving \pc{6.6} more problems than a baseline configuration.

Notably, the RankNet-based approach is sufficiently generic to be applied to other decision points in \gls{atping}.
In the past, I successfully applied it to symbol precedence recommendation \cite{DBLP:conf/cade/Bartek021}.

In my presentation, I will explain the design of RankNet and its generalization DirectRanker \cite{DBLP:conf/pkdd/KoppelSWPK019},
and describe how I applied it to clause selection and symbol precedence recommendation.

\bibliographystyle{plain}
\bibliography{main}

\end{document}
